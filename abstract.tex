%% -*- coding: utf-8-unix -*-

\chapter{はじめに}
\label{chap:abstract}

 \section{本書の目的}
 \label{sec:book-purpose}

本書は、沖縄オープンラボラトリで2016年度に実施された「ネットワークテスト
システムプロジェクト」の活動報告書である。本書の目的は、プロジェクトの目
標と実施した活動、活動の結果・成果をまとめ、報告することである。特に、最
終的な成果(物)についての解説だけでなく、そこへいたるまでの検討過程・判断
や選択の基準などについてもとりあげ、「考えかた」についても共有することを
目的としている。

 \section{本プロジェクトの目的}
 \label{sec:pj-purpose}

% TODO: OOD発表資料のp1 : 視点/観点の話をいれること
% OOD発表資料のp.2-3

情報システムを構築し、利用者に提供していくにあたって、システムを構築・運
用する側は、様々な製品や技術をくみあわせてシステムを構成し、サービスを実
現していく必要がある。複数の仮想化技術の導入、拡張性に対する要求とそれを
実現するための自動化技術の発展などにより、システムはより大規模かつ複雑に
なっている。こうした、多数のブラックボックスをくみあわせた複雑なシステム
の構築や運用では、システム構成要素間の連携を把握し、システム内部の一部の
変更によってシステム全体の動作やサービスにどういった影響があるかを判断す
ることは難しくなる。一方、利用者の要求やサービスの変化のサイクルははやく
なっており、システム構築あるいはサービス提供者は、その要求に追従していく
ことが求められる。

そのため、サービス提供者は、サービス全体の動作を判断し、最終的に利用者に
対して価値が提供できているかどうか、サービス利用者の観点でシステムの状況
を判断することがより重要になっていく。本プロジェクトは、特に情報システム
構成要素としての「ネットワーク」に着目し、ネットワークが提供するサービス
の正常性を確認する(テストする)ためのシステム = ネットワークテストシステ
ムの提案、およびその有効性を確認するための実証実験をおこなう。

 \section{前提事項}
 \label{sec:premise}

本書の読者として、ネットワークテストシステムの有効性の判断や、利用・導入
を検討をおこないたいエンジニアを想定している。以下にあげる点については基
礎知識として本書ではとりあげない。

\begin{itemize}
 \item Linuxの基本的な使用方法
       \begin{itemize}
        \item KVM による仮想マシン(VM)管理
        \item パッケージ管理
        \item git の使用方法
        \item Linux Network Namespace および iproute2 ( \code{ip} コマン
              ド)によるNamespace の操作
       \end{itemize}
 \item 基本的なネットワークに関する知識
       \begin{itemize}
        \item TCP/IP および Ethernet/VLAN
        \item Cisco や Juniper の L2/L3 Switch, Firewall の基本的な動作・
              設定
       \end{itemize}
 \item OpenFlow
       \begin{itemize}
        \item OpenFlow/1.0
        \item Trema による OpenFlow Controller の実装
        \item OpenFlow スイッチ (Open vSwitch) の基本的な動作・設定
       \end{itemize}
 \item CI/CD, TDD/BDDに関する基本的な知識
 \item Rubyプログラミング
       \begin{itemize}
        \item RubyEnv をつかった ruby 環境の管理
        \item Bundler をつかったパッケージ管理
        \item Cucumber によるテストシナリオ実装
       \end{itemize}
\end{itemize}

また、本プロジェクトの前身となる、2015年度にOOLで実行したプロジェクト
(「L1patch 応用ネットワークテストシステム」プロジェクト)でとりあつかった
内容については原則とりあげない。2015年度プロジェクトについては、活動報告
書~\cite{l1pjpoc} や技術情報~\cite{l1pjtech} を参照すること。

%%% Local Variables:
%%% mode: yatex
%%% TeX-master: main.tex
%%% End:
