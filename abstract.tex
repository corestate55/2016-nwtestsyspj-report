%% -*- coding: utf-8-unix -*-

\chapter{はじめに}
\label{chap:abstract}

 \section{本書の目的}
 \label{sec:book-purpose}

本書は、沖縄オープンラボラトリ~\cite{ool-web}で2016年度に実施された「ネッ
トワークテストシステムプロジェクト」の活動報告書である。本書の目的は、プ
ロジェクトの目標と実施した活動、活動の結果・成果をまとめ、報告することで
ある。特に、最終的な成果(物)についての解説だけでなく、そこへいたるまでの
検討過程・判断や選択の基準などについてもとりあげ、「考えかた」についても
共有することを目的としている。

 \section{本プロジェクトの目的}
 \label{sec:pj-purpose}

% TODO: OOD発表資料のp1 : 視点/観点の話をいれること
% OOD発表資料のp.2-3

情報システムを構築し、利用者に提供していくにあたって、システムを構築・運
用する側は、様々な製品や技術をくみあわせてシステムを構成し、サービスを実
現していく必要がある。複数の仮想化技術の導入、拡張性に対する要求とそれを
実現するための自動化技術の発展などにより、システムは抽象化されてより柔軟
な制御が可能になった。その反面、大規模かつ複雑にもなっている。こうした、
多数のブラックボックスをくみあわせた複雑なシステムの構築や運用では、シス
テム構成要素間の連携を把握し、システム内部の一部の変更によってシステム全
体の動作やサービスにどういった影響があるかを判断することは難しくなる。一
方で、利用者の要求やサービスの変化のサイクルは速くなっており、システム開
発者あるいはサービス提供者は、その要求に追従していくことが求められる。

このような状況から、今後サービス提供者は、サービス全体の動作を判断し、最
終的に利用者に対して価値が提供できているかどうか、サービス利用者の観点で
システムの状況を判断することがより重要になっていく。本プロジェクトは、サー
ビス変化の迅速性を考えるうえでボトルネックになりがちなネットワークに着目
する。より高速かつ網羅的にネットワークが提供するサービスの正常性を確認す
る(テストする)ためのシステム = ネットワークテストシステムの提案、および
その有効性を確認するための実証実験をおこなう。これによって従来のネットワー
ク構築・運用の迅速性の向上、リスク低減をおこない、CI/CDプロセスを実現さ
せることを目指す。

 \section{関連研究}
 \label{sec:related-research}

テスト・検証の自動化に関する関連研究として、ホワイトボックススイッチと従
来の(レガシーな)ネットワーク機器で構成されたテストベッドネットワークに対
して、OSSを組み合わせて任意の箇所に自動で障害を発生させる検証自動化シス
テム~\cite{wbsw-oss-test-automation}が提案されている。レガシー機器の利用・
end-to-endでのふるまいへの着目など本研究と課題感が近く、テスト環境の自動
構築などテストとして求められる機能一式に対応している。本研究ではテスト対
象環境の自動構築には比重を置いていないこと、テストトラフィックのディスト
リビューションにOpenFlow制御を利用していること、テストシナリオ実装の考え
かたなどに違いがある。

沖縄オープンラボラトリでも、本プロジェクト以外にネットワークテストに関す
る研究がおこなわれている。OF-Patchプロジェクト~\cite{ool-testbedpj}は
OpenFlow制御による物理パッチパネルを実装・OSS公開している。また、それに
よって実際に沖縄オープンラボのテストベッドを構築し、利用者に提供している。
VNFテストシナリオ自動化PJ~\cite{ool-vnftestpj}は複数のネットワーク仮想ア
プライアンスのデプロイパターンや相互接続性に着目しており、ハイパーバイザ
とVNF, VNF同士の接続などさまざまなパターンで接続性や性能測定を網羅的に自
動実行するシステムを構築している。OF-PatchはL1(物理トポロジ)操作に、VNF
テストシナリオ自動化はVNFの利用(組合せ・機能・特性)に焦点をあてている。

ネットワークのテストという観点で利用可能なOSSとしては、
InfraTester~\cite{infratester-github}やToDD~\cite{todd-github}がある。
InfraTesterはネットワークを経由して、特定のサービスの外側からの動作(ふる
まい)をテストすることができる。構成として、定点(InfraTesterサーバ)から見
たときのサービスを観測するかたちになるため、L2/L3のネットワーク(物理的な
分散性)についてはターゲットとしていない。ToDDは、テスト対象ネットワーク
にAgentを配布し、ToDDサーバとメッセージキューによって情報交換をおこなう。
テスト自体はtestletという形でToDDサーバからAgentへおくられ、実行される。
Agentを使うことで、ネットワークの物理的な分散性(distribution)に対応して
おり、拡張性のたかいアーキテクチャをもっている。

本PoCではターゲットとしていない(\ref{sec:discuss-network-test}節参照)が、
テスト対象となるネットワーク機器の操作をおこなうための仕組みに関する近年
の動向について簡単に触れておく。従来のネットワーク機器(CLI)を抽象化する
ためのOSS開発~\cite{netmiko-github,napalm-github}は引きつづき活発におこな
われている\footnote{本研究では相当する機能について、よりシンプルで簡易的
なツールを使用している(\ref{sec:expectacle}節参照)。}。また、Ansibleはバー
ジョン2.2よりネットワーク機器の操作に対応~\cite{ansible-22-news}しており、
利用事例が増えていくことが予想される。技術動向としては、Google等のネット
ワークオペレータによってベンダ中立なネットワーク機器インタフェース(API)
策定の活動がおこなわれている(OpenConfig~\cite{openconfig})。OpenConfigに
ついてはJuniper, Cisco, Aristaなどの商用製品による採
用~\cite{openconfig-news}もはじまっている。

 \section{前提事項}
 \label{sec:premise}

本書の読者として、ネットワークテストシステムの有効性の判断や、利用・導入
を検討をおこないたいエンジニアを想定している。以下にあげる点については基
礎知識として本書ではとりあげない。

\begin{itemize}
 \item Linuxの基本的な使用方法
       \begin{itemize}
        \item KVM による仮想マシン(VM)の操作/管理
        \item 必要なツールやソフトウェアのインストール/パッケージ管理
        \item git の使用方法
        \item Linux Network Namespace および iproute2 ( \code{ip} コマン
              ド)によるNamespace の操作
       \end{itemize}
 \item 基本的なネットワークに関する知識
       \begin{itemize}
        \item TCP/IP および Ethernet/VLAN
        \item Cisco や Juniper の L2/L3 Switch, Firewall の基本的な動作・
              設定
       \end{itemize}
 \item OpenFlow
       \begin{itemize}
        \item OpenFlow/1.0
        \item Trema による OpenFlow Controller の実装
        \item OpenFlow スイッチ (Open vSwitch) の基本的な動作・設定
       \end{itemize}
 \item Rubyプログラミング
       \begin{itemize}
        \item RubyEnv をつかった ruby 環境の管理
        \item Bundler をつかったパッケージ管理
        \item Cucumber によるテストシナリオ実装
       \end{itemize}
\end{itemize}

また、本プロジェクトの前身となる、2015年度にOOLで実行したプロジェクト
(「L1patch 応用ネットワークテストシステム」プロジェクト、通称\lopj)でと
りあつかった内容については原則とりあげない。2015年度プロジェクトについて
は、\lopjpoc や \lopjtech を参照すること。

%%% Local Variables:
%%% mode: yatex
%%% TeX-master: "main.tex"
%%% End:
