%% -*- coding: utf-8-unix -*-

\chapter{おわりに}

 \section{まとめ}
 \label{sec:summary}

複数の機器や技術の整合性をとらなければいけないネットワークは、複雑になり
がちで設定変更や障害の影響範囲を予測しにくい。しかしサービス提供者は、利
用者の要求変化に応じて、安定した通信サービスを迅速かつ柔軟に提供していく
ことが求められる。人によるレビューや検証作業は、検討可能な範囲や許容され
るリードタイムに限界がある。そこで、ネットワークが機械的・自動的にテスト
できること、テストによってネットワークが設計通りの機能を提供していること
が確認可能になることが重要となる(\ref{chap:problem-setting}章)。本プロジェ
クトでは特に、ネットワークの「ふるまい」に注目して(\ref{chap:approach}章)、
物理ネットワークの自動テストをおこなうシステムを開発した
(\ref{chap:nettester-design}章・\ref{chap:nettester-usage}章)。また、い
くつかのユースケースについて実際にシナリオテストの実装をおこなった
(\ref{chap:poc-target-design}章・\ref{chap:poc-scenario-dev}章)。

テストシナリオ実装においては、ネットワークが定常状態に状況でのend-to-end
通信を「静的なふるまいのテスト」とした。また、リンク障害による通信経路の
切替がおきてネットワークの状態が変化する状況でのend-to-end通信を「動的な
ふるまいのテスト」とした。これらのテスト実装によって、まず定常状態でのア
プリケーションレベルでの通信試験を実現した。さらに、障害試験のように物理
構成の操作が必要で、従来は人手で作業しなければいけなかった試験についても、
自動化できるようになった。また、テストスイートの機能を利用して、手作業で
の実施が難しい高度なテスト(回帰テストやランダムテスト)の実行が可能になる
ことを検証した。本書ではこれらの物理ネットワーク試験自動化における実装例
と、実装の際に発生した課題に対する対処などについて解説している。

本プロジェクトで実証したユースケースは単純化したものになっているが、物理
ネットワークでおこなうテストとして通常実施される、基本的な作業を自動化し
たものである。本プロジェクトで整備した「ふるまい」レベルでのネットワーク
テスト機能により、\ref{sec:discuss-network-test}節で示したネットワークテ
スト自動化の基本機能は一通りそろえることができた。

 \section{今後の課題}
 \label{sec:future-work}
 % - 今後想定される運用・開発のプロセス
 % - 「できないこと」のテスト
 % - 実用トライアル

現時点では、架空の企業ネットワークを想定した小規模かつシンプルなネットワー
クでPoCをおこなっている。しかし、実際に実環境・実運用で使用する場合は、
より複雑な環境・要件に対応していくことが求められる。

\ref{sec:summary}節で示したように、ネットワークテストの基本機能は整備で
きた。今後は実環境での利用を想定した実用トライアルにむけて、活動を進めて
いく。そのなかで、\ref{sec:desired-and-target}節で示したようなネットワー
クにおけるCI/CDプロセスの確立、ツールチェーンのありかたや実装などについ
ての検討をおこなう。また、開発したツールについて、実際の問題解決をしてい
くために必要な機能・非機能面の機能強化を進める。


%%% Local Variables:
%%% mode: yatex
%%% TeX-master: "main.tex"
%%% End:
