%% -*- coding: utf-8-unix -*-

\chapter{おわりに}

 \section{まとめ}
 \label{sec:summary}

複数の機器や技術の整合性をとらなければいけないネットワークは、複雑になり
がちで設定変更や障害の影響範囲を予測しにくい。しかしサービス提供者は、利
用者の要求変化に応じて、安定した通信サービスを迅速かつ柔軟に提供していく
ことが求められる。そのためには、検討可能範囲や許容されるリードタイムに限
界がある人によるレビューではなく、ネットワークが機械的・自動的にテストで
きること、テストによってネットワークが設計通りの機能を提供していることを
確認できることが重要となる(\ref{chap:problem-setting}章)。本プロジェクト
では特に、ネットワークの「ふるまい」に注目して(\ref{chap:approach}章)、
物理ネットワークの自動テストをおこなうシステムを開発し
(\ref{chap:nettester-design}章・\ref{chap:nettester-usage}章)、いくつか
のユースケースについて実際にシナリオテストの実装をおこなった
(\ref{chap:poc-target-design}章)。

テストシナリオ実装においては、ネットワークが定常状態に状況でのend-to-end
通信を「静的なふるまいのテスト」、リンク障害による通信経路切替がおきてネッ
トワークの状態が変化する状況でのend-to-end通信を「動的なふるまいのテスト」
とした。これらのテスト実装によって、定常状態でのアプリケーションレベルで
の通信試験だけでなく、障害試験のように物理構成操作が必要で従来は人手で作
業しなければいけなかった試験についても自動化ができるようになった。本書で
はこれらの物理ネットワーク試験自動化における実装例と実装の際に発生した課
題に対する対処などについて解説している。

本プロジェクトで実証したユースケースは単純化したものになっているが、物理
ネットワークでおこなうテストとして通常実施される、基本的な作業を自動化し
たものである。本プロジェクトで整備した「ふるまい」レベルでのネットワーク
テスト機能により、\ref{sec:discuss-network-test}節で示したネットワークテ
スト自動化のための基本機能は一通りそろえることができたと考える。

 \section{今後の課題}
 \label{sec:future-work}
 % - 今後想定される運用・開発のプロセス
 % - 「できないこと」のテスト
 % - 実用トライアル

現時点では、架空の企業ネットワークを想定した小規模かつシンプルなネットワー
クでのPoCを行なっている。しかし、実際に実環境・実運用で使用する場合は、
より複雑な環境・要件に対応していくことが求められる。

\ref{sec:summary}節で示したように、ネットワークテストの基本機能はおおむ
ね整備できた。今後は実環境での利用を想定した実用トライアルにむけた活動を
進めていく。そのなかで、\ref{sec:desired-and-target}節で示したようなネッ
トワークにおけるCI/CDプロセスの確立、そのためのツールチェーンのありかた
や実装などについての検討をおこなう。また、開発したツールについて、実際の
問題解決のために必要な機能・非機能面での機能強化を進める。


%%% Local Variables:
%%% mode: yatex
%%% TeX-master: "main.tex"
%%% End:
