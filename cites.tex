%% -*- coding: utf-8-unix -*-

\begin{thebibliography}{99}
 \bibitem{ool-web}
         ``一般社団法人沖縄オープンラボラトリ \verb+|+ Okinawa Open Laboratory'',
         \url{http://www.okinawaopenlabs.org/}
 \bibitem{ool-l1pj-web}
         ``L1Patch応用NWテストシステム プロジェクト'',
         \url{http://www.okinawaopenlabs.org/archives/research2014/150410}
 \bibitem{l1pjpoc}
         ``L1patch応用ネットワークテストシステム 試験結果レポート''.
         \url{https://github.com/oolorg/ool-l1patch-dev/blob/master/report/l1pj-poc-report-20151114.pdf}.
         2015.
 \bibitem{l1pjtech}
         ``L1patch応用ネットワークテストシステム 技術レポート''.
         \url{https://github.com/oolorg/ool-l1patch-dev/blob/master/report/l1pj-tech-report-20151114.pdf}.
         2015.
 \bibitem{ool-testbedpj}
         ``OF-Patch拡張 プロジェクト'',
         \url{https://www.okinawaopenlabs.org/archives/research2014/141006}
 \bibitem{ool-vnftestpj}
         田部英樹, ``VNFテストシナリオ自動化PJについて'',
         \url{https://www.okinawaopenlabs.org/wp/wp-content/uploads/8_VNF%E3%83%86%E3%82%B9%E3%83%88%E3%82%B7%E3%83%8A%E3%83%AA%E3%82%AA%E8%87%AA%E5%8B%95%E5%8C%96%E3%83%97%E3%83%AD%E3%82%B8%E3%82%A7%E3%82%AF%E3%83%88%E3%81%AB%E3%81%A4%E3%81%84%E3%81%A6.pdf}, 2015.
 \bibitem{wbsw-oss-test-automation}
         渋谷惠美, 川上秀彦, 長谷川輝之, 山口弘純, ``ホワイトボックスス
         イッチとOSSを活用したネットワークに対する検証自動化システム設計
         に関する一提案'', 信学技報, vol. 116, no. 111, NS2016-34,
         pp. 35-40, 2016年6月.
 \bibitem{needlework-web}
         ``NEEDLEWORK \verb+|+ AP Communications'',
         \url{http://www.ap-com.co.jp/ja/needlework/}
 \bibitem{needlework-slide}
         鈴木飛鳥,``FWのポリシーテストを自動化してみた'',
         \url{https://www.slideshare.net/tanksuzuki/fw-59102915},
         NetOpsCoding\#2, 2016.
 \bibitem{infratester-github}
         ``Infrataster by ryotarai'',
         \url{http://infrataster.net/}
 \bibitem{todd-github}
         ``toddproject/todd: A highly extensible framework for distributed capacity and connectivity testing (Testing on Demand....Distributed!)'',
         \url{https://github.com/toddproject/todd}
 \bibitem{todd-blog}
         ``The Power of Test-Driven Network Automation'',
         \url{https://keepingitclassless.net/2016/03/test-driven-network-automation/}
 \bibitem{openvnet-web}
         ``OpenVNet'', \url{https://openvnet.org/}
 \bibitem{openvnet-slide}
         山崎泰宏, ``OpenVNet - SDNで物理ネットワークアプライアンスをプ
         ロビジョニングしよう'',
         \url{https://www.slideshare.net/yasuhiro_yamazaki/openvnet-sdn},
         ネットワークプログラマビリティ勉強会\#10, 2016.
 \bibitem{network-testing-sdn-atmarkit}
         村木暢哉, ``SDNで始めるネットワークの運用改善(2):SDNで物理ネッ
         トワークのテストを楽にする方法 (1/4) - @IT'',
         \url{http://www.atmarkit.co.jp/ait/articles/1612/27/news014.html},
         @IT, 2017.
 \bibitem{netmiko-github}
         ``ktbyers/netmiko: Multi-vendor library to simplify Paramiko SSH connections to network devices'',
         \url{https://github.com/ktbyers/netmiko}
 \bibitem{trigger-github}
         ``trigger/trigger: Trigger is a robust network automation toolkit written in Python that was designed for interfacing with network devices.'',
         \url{https://github.com/trigger/trigger}
 \bibitem{napalm-github}
         ``napalm-automation/napalm: Network Automation and Programmability Abstraction Layer with Multivendor support'',
         \url{https://github.com/napalm-automation/napalm}
 \bibitem{ansible-web}
         ``Ansible is Simple IT Automation'',
         \url{https://www.ansible.com/}
 \bibitem{ansible-22-news}
         ``Ansible 2.2、コンテナ、ネットワーク、クラウドサービス向けの新自動化機能を提供'',
         \url{https://www.redhat.com/ja/about/rhjapan-press-releases/ansible-22-delivers-new-automation-capabilities-containers-networks-and-cloud-services}
 \bibitem{openconfig}
         ``OpenConfig - Home'',
         \url{http://www.openconfig.net/}
 \bibitem{openconfig-news}
         ``OpenConfig - News'',
         \url{http://www.openconfig.net/news/}
 \bibitem{bdd-cycle-figref}
         ``Should TDD and BDD be used in conjunction? - Stack Overflow'',
         \url{http://stackoverflow.com/questions/33746804/should-tdd-and-bdd-be-used-in-conjunction}
 \bibitem{nettester}
         ``net-tester/net-tester: 物理ネットワークのための受け入れテストツール'',
         \url{https://github.com/net-tester/net-tester}.
 \bibitem{nettester-ex}
         ``net-tester/examples: NetTester用のサンプルテストコード'',
         \url{https://github.com/net-tester/examples}
 \bibitem{nettester-demo-movie}
         ``NetTesterでテスト自動化!~Network Test System Project~ - YouTube'',
         \url{https://www.youtube.com/watch?v=C7z3aaWgsf4}
 \bibitem{wikipedia-bdd}
         ``ビヘイビア駆動開発 - Wikipedia'',
         \url{https://ja.wikipedia.org/wiki/%E3%83%93%E3%83%98%E3%82%A4%E3%83%93%E3%82%A2%E9%A7%86%E5%8B%95%E9%96%8B%E7%99%BA}.
 \bibitem{nettester-pry}
         ``NetTesterでad-hocなテスト作業を拡張する - Qiita'',
         \url{http://qiita.com/corestate55/items/d6a8cdc03de09a46877c}
 \bibitem{ovs-vswitchd-doc}
         ``ovs-vswitchd.conf.db.5.txt'',
         \url{http://openvswitch.org/support/dist-docs/ovs-vswitchd.conf.db.5.txt}
 \bibitem{ovs-backoff-doc}
         ``[ovs-discuss] Forcing Switch to Reconnect'',
         \url{https://mail.openvswitch.org/pipermail/ovs-discuss/2012-January/006368.html}
 \bibitem{iproute2-doc}
         ``ip-netns(8) - Linux manual page'',
         \url{http://man7.org/linux/man-pages/man8/ip-netns.8.html}
 \bibitem{build-essential-doc}
         ``Ubuntu – xenial の build-essential パッケージに関する詳細'',
         \url{http://packages.ubuntu.com/ja/xenial/build-essential}
 \bibitem{rbenv}
         ``rbenv/rbenv: Groom your app’s Ruby environment'',
         \url{https://github.com/rbenv/rbenv}
 \bibitem{pry}
         ``pry/pry: An IRB alternative and runtime developer console'',
         \url{https://github.com/pry/pry}
 \bibitem{net-tester-pr7} ``implement a work-around of tcp checksum
         error in vlan trunk · Pull Request \#7 ·
         net-tester/net-tester'',
         \url{https://github.com/net-tester/net-tester/pull/7}
 \bibitem{rspec-book} David Chelimsky, Dave Astels , Zach Dennis ``The
         RSpec Book'', 株式会社クイープ(訳), 角谷信太郎, 豊田祐司(監修),
         Professional Ruby Series, 翔泳社, 2012.
 \bibitem{spiral-workflow}
         ``クライアントの要望にこたえるWebサービス開発 ~「らせん型ワークフロー」のススメ~''.
         \url{http://www.slideshare.net/mayuco/css-nite-in-sapporo-vol5-14085124}
 \bibitem{j3g14-packet-forwarding}
         吉田友哉, 松崎吉伸,
         ``幸せなパケット転送 ○○編'',
         \url{https://www.janog.gr.jp/meeting/janog14/src/janog14-yoshida-maz.pdf},
         Janog14, 2004.
 \bibitem{yoyodyne}
         ``ヨーヨーダイン - Wikipedia'',
         \url{https://ja.wikipedia.org/wiki/\%E3\%83\%A8\%E3\%83\%BC\%E3\%83\%A8\%E3\%83\%BC\%E3\%83\%80\%E3\%82\%A4\%E3\%83\%B3}
 \bibitem{h25nwsp}
         IPA 独立行政法人 情報処理推進機構,
         ``平成25年度 秋季 ネットワークスペシャリスト試験 午後I問題'',
         \url{https://www.jitec.ipa.go.jp/1_04hanni_sukiru/mondai_kaitou_2013h25_2/2013h25a_nw_pm1_qs.pdf}, 2013.
 \bibitem{cucumber}
         ``Cucumber'',
         \url{https://cucumber.io/}
 \bibitem{expectacle}
         ``expectacle'',
         \url{https://rubygems.org/gems/expectacle}
 \bibitem{trema-logger-doc}
         ``Logger - Trema - Relish'',
         \url{http://www.relishapp.com/trema/trema/docs/logger}
 \bibitem{pio}
         ``trema/pio: Packet parser and generator in Ruby'',
         \url{https://github.com/trema/pio}
 \bibitem{phut}
         ``trema/phut: Virtual network in seconds'',
         \url{https://github.com/trema/phut}
 \bibitem{of10spec}
         ``OpenFlow Switch Specification Version 1.0.0'',
         \url{http://archive.openflow.org/documents/openflow-spec-v1.0.0.pdf}
 \bibitem{pio-pr320} ``Fix the read\_length of PacketIn\#raw\_data ·
         Pull Request \#320 · trema/pio'',
         \url{https://github.com/trema/pio/pull/320}
 \bibitem{trema-pr433} ``Debug print messages that are sent and received
         · Pull Request \#433 · trema/trema'',
         \url{https://github.com/trema/trema/pull/433}
 \bibitem{screenos-releases} ``ScreenOS Release 6.3.0 Software
         Documentation for SSG, ISG, and NetScreen Series - Technical
         Documentation - Support - Juniper Networks'',
         \url{http://www.juniper.net/techpubs/en_US/screenos6.3.0/information-products/pathway-pages/screenos/index.html}
 \bibitem{test-double} ``テストダブル - Wikipedia'',
         \url{https://ja.wikipedia.org/wiki/%E3%83%86%E3%82%B9%E3%83%88%E3%83%80%E3%83%96%E3%83%AB}
 \bibitem{factory-girl} ``thoughtbot/factory\_girl: A library for setting
         up Ruby objects as test data.'',
         \url{https://github.com/thoughtbot/factory_girl}
\end{thebibliography}

%%% Local Variables:
%%% mode: yatex
%%% TeX-master: "main.tex"
%%% End:
