%% -*- coding: utf-8-unix -*-

\chapter{用語}
\label{cpt:termdef}

表\ref{tbl:termdef} に本書で使用している用語・略語の一覧を示す。一般的な
用語については特に解説をくわえない。

 \begin{longtable}{p{8em}|p{8em}|c|c|p{16em}}
  \caption{用語定義}
  \label{tbl:termdef}
  \\
  \hline
  用語 & 英語表記 & 略語 & 分類 & 意味\\ \hline
  \hline
  \endhead
  沖縄オープンラボ & Okinawa Open Laboratory & OOL & 一般 & (略語定義) 沖縄オープンラボラトリ~\cite{ool-web} \\ \hline
  ネットワーク & Network & NW & 一般 & (略語定義) \\ \hline
  ファイアウォール & Firewall & FW & 一般 & (略語定義) \\ \hline
  ロードバランサ & Load Balancer & LB & 一般 & (略語定義) \\ \hline
  & Virtual Private Network & VPN & 一般 & (略語定義) \\ \hline
  L1パッチ & L1patch & & プロジェクト & 一般的なネットワーク用パッチパネル(物理結線を集中して付け替えるための機構)。本書では「ネットワーク用パッチパネルと同等のことをおこなえるシステム」として使う。 \\ \hline
  & OpenFlow & OF & 一般 & (略語定義) \\ \hline
  & OpenFlow Switch & OFS & 一般 & (略語定義) \\ \hline
  & OpenFlow Controller & OFC & 一般 & (略語定義) \\ \hline
  テスト対象ネットワーク & Target/Testee Network & & 一般 & 動作確認を行いたいネットワーク。複数のネットワーク機器から構成される。\\ \hline
  テスター & Tester & & プロジェクト & 本書で単に「テスター」とした場合はネットワークテスター(ネットワークテストシステム): NetTester のことを指す。\\ \hline
  テスト対象機器 & Device Under Test & DUT & 一般 & テスト対象となる個々の機器。テスト対象ネットワークを構成するいずれかひとつの機器。\\ \hline
  & Open vSwitch & OVS & 一般 & Linux上で動作するL2スイッチ/OFS実装。 \\ \hline
  & NetTester & & プロジェクト & 本プロジェクトで開発したネットワークテストツール。 \\ \hline
  & Cucumber & & 一般 & BDDテストツール/テストシナリオ記述言語。 \\ \hline
  継続的インテグレーション & Continuous Integration & CI & 一般 & (略語定義) \\ \hline
  継続的デリバリ & Continuous Delivery & CD & 一般 & (略語定義) \\ \hline
  ふるまい駆動開発 & Behavior Driven Development & BDD & 一般 & (略語定義) \\ \hline
  テスト駆動開発 & Test Driven Devleopment & TDD & 一般 & (略語定義) \\ \hline
 \end{longtable}

%%% Local Variables:
%%% mode: yatex
%%% TeX-master: "main.tex"
%%% End:
