%% -*- coding: utf-8-unix -*-

\chapter{ネットワークのテストシナリオ実装}

\section{自動テスト基礎}

\begin{itemize}
 \item TDD/BDDとBDDツールとしてのCucumber
 \item Narrative
\end{itemize}

\section{ネットワークのテストシナリオ}

\subsection{テストシナリオの概要}

\begin{itemize}
 \item ネットワークテストを書く上での検討点、今回のプロジェクトでの方針・決めごと
 \item なぜそうきめたのか?
\end{itemize}

\subsection{ステップ実装上の工夫}
\begin{itemize}
 \item ニセ○○サーバとステップ実装 – NetTester \url{https://3.basecamp.com/3088280/buckets/867009/documents/216490375}
 \item コマンドをバックグラウンド実行 – NetTester \url{https://3.basecamp.com/3088280/buckets/867009/documents/216399643}
 \item step内でのバックグラウンドコマンド実行 – NetTester \url{https://3.basecamp.com/3088280/buckets/867009/todos/202691188}
 \item factory\_girl で仮想ホストを作る – NetTester \url{https://3.basecamp.com/3088280/buckets/867009/documents/210831650}
\end{itemize}

%%% Local Variables:
%%% mode: yatex
%%% TeX-master: "main.tex"
%%% End:
